\documentclass[12pt]{article}
\usepackage{amsfonts,amsmath,amssymb,graphicx,url}

% Old Stuff
%%\oddsidemargin=0.15in
%%\evensidemargin=0.15in
%%\topmargin=-.5in
%%\textheight=9in
%%\textwidth=6.25in

\setlength{\oddsidemargin}{.25in}
\setlength{\evensidemargin}{.25in}
\setlength{\textwidth}{6.25in}
\setlength{\topmargin}{-0.4in}
\setlength{\textheight}{8.5in}

\newcommand{\heading}[5]{
   \renewcommand{\thepage}{#1-\arabic{page}}
   \noindent
   \begin{center}
   \framebox{
      \vbox{
    \hbox to 6.2in { {\bf CS390 Computational Game Theory and Mechanism Design}
         \hfill #2 }
       \vspace{4mm}
       \hbox to 6.2in { {\Large \hfill #5  \hfill} }
       \vspace{2mm}
       \hbox to 6.2in { {\it #3 \hfill #4} }
      }
   }
   \end{center}
   \vspace*{4mm}
}

\newcommand{\handout}[3]{\heading{#1}{#2}{2012 ACM Class 5120309027 Huang Zen}{}{#3}}

\setlength{\parindent}{0in}
\setlength{\parskip}{0.1in}

\newenvironment{proof}{\noindent{\em Proof.} \hspace*{1mm}}{
\hspace*{\fill} $\Box$ }

\begin{document}
\handout{1}{July 5, 2013}{Problem Set 1}

\paragraph{Problem 1} (No collaborator.)
\begin{itemize}
\item[(a)]
There is particular no pure Nash equilibrium. However there is a unique mixed
Nash equilibrium where the two players both choose their three actions
in equal probability.
\\
\begin{proof}\\
First we show that the mixed strategy profile
$\big((\frac{1}{3},\frac{1}{3},\frac{1}{3}),(\frac{1}{3},\frac{1}{3},\frac{1}{3})\big)$
is a NE. Given one's strategy with
$\delta_1=(\frac{1}{3},\frac{1}{3},\frac{1}{3})$, then the utility of
another player would be $u_2 = \sum(\frac{1}{3}p_i - \frac{1}{3}p_i) =
0$, so any mixed strategy for player2 is a best response. Then
particularly the one with $(\frac{1}{3}, \frac{1}{3}, \frac{1}{3})$ is
a best mixed response, for both the players.
\\
Then we show that there's only one such mixed NE. Since the R-P-C game
is Zero-Sum, when one player's strategy is not $(\frac{1}{3},
\frac{1}{3}, \frac{1}{3})$, the other player would be able to choose
his strategy so that $u_2>0$, for instance put all probability in the
action which beats the other's most likely action. And while the same
for player1 can choose $u_1>0$, since $u_1+u_2=0$, there can't be
another mixed NE.
\end{proof}

\item[(b)]
A mixed NE is
$\big((\frac{2}{3},\frac{1}{3}),(\frac{1}{3},\frac{2}{3})\big)$.
\\
\begin{proof}\\
Given player1's strategy with $\sigma_1 = (\frac{2}{3},
\frac{1}{3})$. Let $\sigma_2$ be $(p, q)$. Then $u_2 = 1\frac{2}{3}p +
2\frac{1}{3}q = \frac{2}{3}(p + q) = \frac{2}{3}$, so any strategy of
player 2 is a best mixed response, particularly the one with $\sigma_2
= (\frac{1}{3}, \frac{2}{3})$ is. By symmetry, this is true for
player 2 either.
\end{proof}

\item[(c)]
Obviously there are two pure NE in BoS with both B or both S. As (b) shows there's another mixed
NE. We prove this is the only mixed NE so that we have in total 3
equilibrium.
\\
\begin{proof}\\
Suppose player1 chooses some strategy $\sigma_1=(p, q)$. Suppose
$p>\frac{2}{3}$. Then the best response for player2 would be
$\sigma_2=(1,0)$. In that case the best response for player1 is
$\sigma_1=(1,0)$, which is a pure NE counted. By symmetry, this is
true for $p<\frac{2}{3}$. So
$\big((\frac{2}{3},\frac{1}{3}),(\frac{1}{3},\frac{2}{3})\big)$ is the
only mixed NE.
\end{proof}

\item[(d)]
Two players each pick a positive number. The utility for each would be
the number they choose. Obviously there's no NE since each can always
find a better strategy by changing to a bigger number.

\end{itemize}

\bigskip

\paragraph{Problem 2} (No collaborator.)
\begin{itemize}
\item[(a)]
Formulate a first price auction as a normal form game.

\begin{align*}
N &= \{1, 2, 3, \cdots, n\} \\
S &= N_+^n \\
u_i &= \left\{ \begin{array}{ll}
v_i - s_{max}  & s_i = s_{max} \text{ and } i < j
\text{ if } s_j = s_{max}\\
0 & \text{otherwise}
\end{array} \right.
\end{align*}

Then if there is a NE, player 1 must  obtains the object.
\\
\begin{proof}\\
If not, let player p$(p>1)$ be the one obtains the object. Then we
have $s_1<s_k\leq v_k<v_1$. Then player can change his bid to at
least $v_k$ to win the auction with non-zero utility.
\end{proof}
\\
Then we find all the NE. As is shown before, any NE must have
$s_{max}=s_1$. Also $s_{max}$ can't be lower than $v_2$, otherwise
player 2 can change his bit to $v_2$ to rise the $s_{max}$.
So any NE $S=(s_1,s_2,\cdots,s_n)$ follows:
\begin{itemize}
\item $v_1 \geq s_1 \geq v_2$
\item $\forall j, s_j \leq s_1$
\end{itemize}

\item[(b)]
Formulate weak dominance:

A strategy $s_i$ weakly dominate $s_i'$ if $$\forall s_{-i}, u_i(s_i,
s_{-i}) \geq u_i(s_i', s_{-i})$$ and a strategy is weakly dominant if
it weakly dominate any other strategies.
\\
In a second price auction the bid $v_i$ of any player i is a weakly
dominant strategy.
\begin{proof}\\
We will show for player i, bidding $v_i$ is a weak dominance. Let
$s_i'$ be another bid.In the case $\displaystyle \max_{\substack{j\neq
    i}}s_j \geq v_i$, then with $s_i'$, player i either loses the
object or get non-positive utility. In another case $\displaystyle
\max_{\substack{j\neq i}}s_j<v_i$, player i either loses the object or
wins with the same utility as $s_i$ does.
\end{proof}\\
 Finally, let's consider a equilibriam in which the winner is not
 player 1. Let $N=3$.
$$v_1 = 3, v_2 = 2, v_3 = 1$$
$$s_1 = 1, s_2 = 100, s_3 = 2$$
It is indeed a NE and in this case player 2 wins the object.
\end{itemize}

\bigskip

\paragraph{Problem 3} (No collaborator.)
In the first iteration, all numbers in $[34,100]$ will be eliminated,
since they cannot be 
$\frac{1}{3}$ of the avarage number. Similarly, the following
sequence of elimination will
be $[12,33],[5,12],[3,5],[2,3]$ and what is
left is $1$.


\bibliographystyle{agsm}

\begin{thebibliography}{99}

\bibitem{OR94}{M. J. Osborne and A. Rubinstein. {\em A course in game theory.} MIT Press, 1994.}

\bibitem{NRTV07}{N. Nisan, T. Roughgarden, E. Tardos, and V. Vazirani (eds). {\em Algorithmic game theory.} Cambridge University Press, 2007. (Available at \url{http://www.cambridge.org/journals/nisan/downloads/Nisan_Non-printable.pdf}.)}

\end{thebibliography}

\end{document}








