\documentclass[12pt]{article}
\usepackage{amsfonts,amsmath,amssymb,graphicx,relsize,url,charter}

\setlength{\oddsidemargin}{.25in}
\setlength{\evensidemargin}{.25in}
\setlength{\textwidth}{6.25in}
\setlength{\topmargin}{-0.4in}
\setlength{\textheight}{8.5in}

\newcommand{\heading}[5]{
   \renewcommand{\thepage}{#1-\arabic{page}}
   \noindent
   \begin{center}
   \framebox{
      \vbox{
    \hbox to 6.2in { {\bf CS390 Computational Game Theory and Mechanism Design}
     	 \hfill #2 }
       \vspace{4mm}
       \hbox to 6.2in { {\Large \hfill #5  \hfill} }
       \vspace{2mm}
       \hbox to 6.2in { {\it #3 \hfill #4} }
      }
   }
   \end{center}
   \vspace*{4mm}
}

\newcommand{\handout}[3]{\heading{#1}{#2}{2012 ACM Class 5120309027 Huang Zen}{}{#3}}

\setlength{\parindent}{0in}
\setlength{\parskip}{0.1in}

\newenvironment{proof}{\noindent{\em Proof.} \hspace*{1mm}}{
\hspace*{\fill} $\Box$ }

\newtheorem{definition}{Definition}
\newtheorem{remark}{Remark}
\newtheorem{theorem}{Theorem}
\newtheorem{lemma}[theorem]{Lemma}
\newtheorem{corollary}[theorem]{Corollary}
\newtheorem{proposition}[theorem]{Proposition}
\newtheorem{claim}[theorem]{Claim}
\newtheorem{observation}{Observation}

\begin{document}
\handout{1}{July 22, 2013}{Problem Set 6}

\paragraph{Problem 1} (No collaborator.)
\\
\begin{proof}
\begin{enumerate}
\item[1.]
From
$
\left\{\begin{array}{l}
u_1(\theta,a)>u_1(\theta,c)\\
u_1(\theta',c)>u_1(\theta',a)
\end{array}\right .
$
we have $L(\theta,a)\supset L(\theta',a)$.
By symmetry we also have $L(\theta',b)\supset L(\theta,b)$.
So $f$ is monotone.

\item[2.]
$c\notin f(\theta),c\notin f(\theta')$, so $f$ do not satisfy NVP.

\item[3.]
Suppose there exists a mechanism $M$ which fully Nash-implements $f$.
From $o(\theta,pNE(M))=f(\theta)$,
we have there exist a column that the outcomes are all a.
and from $o(\theta',pNE(M))=f(\theta')$,
then there exist a row that the outcomes are all b.
That leads to $a=b$, what is contradiction.
\end{enumerate}
\end{proof}

\bigskip

\paragraph{Problem 2} (No collaborator.)
\\
\begin{claim}
\end{claim}
\begin{proof}

Let $\displaystyle A^{OPT}=\underset{A'}{argmax}\sum_{i}v_i(A_i')$, 
suppose $A_i^{OPT}=\emptyset$. Then
$$P_i=\underset{A' }{\max}\sum_{j\not = i}v_j(A_j')-\sum_{j \not = i}v_j(A_j^{OPT})=0$$
So that we have claim 1.

\end{proof}

\begin{claim}
\end{claim}
\begin{proof}

Let $\displaystyle A^{OPT}=\underset{A'}{argmax}\sum_{i}v_i(A_i')$, 
we have
$$P_i=\underset{A' }{\max}\sum_{j\not = i}v_j(A_j')-\sum_{j \not = i}v_j(A_j^{OPT})$$
$\forall i\in N, \theta_i\in \Theta_i$, and $v_{-i}\in \Theta_{-i}$,  where $v_i=\theta_i$, we have
\begin{align*}
u_i(\theta_i,VCG(\theta_i,v_{-i}))&=\theta_i(A_i)-P_i\\
&=v_i(A_i)-P_i \\
&=v_i(A_i)+\sum_{j\neq i}v_j(A_j^{OPT})-\underset{A'}{\max}\sum_{j\neq i}v_j(A_j')\\
&=\sum_{j}v_j(A_j^{OPT})-\underset{A'}{\max}\sum_{j\neq i}v_j(A_j')\\
&\geq 0 
\end{align*}

\end{proof}

\bigskip

\paragraph{Problem 3} (No collaborator.)
\\

\begin{enumerate}

\item[(a)]

\begin{proof}
First we show that $M$ implements $F$.
$$
M(\theta)=\left\{\begin{array}{ll}
(0,p) & \sum_i\theta_i<c\\ 
(1,p) & \sum_i\theta_i\geq c
	\end{array}\right .\subseteq F(\theta)
$$
So  $M$ implements $F$. 
\\
Next we show that $M$ is DST, which means $\forall\theta,i$, $\theta_i$ is a dominant strategy. 
$\forall v_{-i}$
\begin{enumerate}
\item[(1)] If $\displaystyle\sum_{j\neq i}v_j+\theta_i <c$

\begin{enumerate}
\item[i.]
If $\displaystyle\sum_{j}v_j<c$, then $u_i(\theta_i,M(\theta_i,v_{-i}))=u_i(\theta_i,M(v_i,v_{-i}))=0$.
\item[ii.]
If $\displaystyle\sum_{j}v_j\geq c$, then 
$\left\{\begin{array}{l}
		u_i(\theta_i,M(\theta_i,v_{-i}))=0\\
		u_i(\theta_i,M(v_i,v_{-i}))=\sum_{j\neq i}v_j+\theta_i-c\leq 0
\end{array}\right .
$.
\end{enumerate}

\item[(2)] If $\displaystyle\sum_{j\neq i}v_j+\theta_i\geq c$
\begin{enumerate}
\item[i.]
If $\displaystyle\sum_{j}v_j<c$, then
$\left\{\begin{array}{l}
		u_i(\theta_i,M(\theta_i,v_{-i}))=\sum_{j\neq i}v_j+\theta_i-c\geq 0\\
		u_i(\theta_i,M(v_i,v_{-i}))=0
\end{array}\right .
$.
\item[ii.]
If $\displaystyle\sum_{j}v_j\geq c$, then
$u_i(\theta_i,M(\theta_i,v_{-i}))=u_i(\theta_i,M(v_i,v_{-i}))=\sum_{j\neq i}v_j+\theta_i-c$.
\end{enumerate}

\end{enumerate}

To all the above cases, 
$u_i(\theta_i,M(\theta_i,v_{-i}))\geq u_i(\theta_i,M(v_i,v_{-i}))$,
so that  $\forall\theta,i$, $\theta_i$ is a dominant strategy.
\\
Thus we have Groves mechanism DST-implements $F$.
\end{proof}

\item[(b)]

$N=2,c=1,\theta_1=\theta_2=10$, the revenue is -18. 

\end{enumerate}


\bibliographystyle{agsm}

\begin{thebibliography}{99}

\bibitem{OR94}{M. J. Osborne and A. Rubinstein. {\em A course in game theory.} MIT Press, 1994.}

\bibitem{NRTV07}{N. Nisan, T. Roughgarden, E. Tardos, and V. Vazirani (eds). {\em Algorithmic game theory.} Cambridge University Press, 2007. (Available at \url{http://www.cambridge.org/journals/nisan/downloads/Nisan_Non-printable.pdf}.)}

\end{thebibliography}

\end{document}








