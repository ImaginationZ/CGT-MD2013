\documentclass[12pt]{article}
\usepackage{amsfonts,amsmath,amssymb,graphicx,relsize,url,charter}

\setlength{\oddsidemargin}{.25in}
\setlength{\evensidemargin}{.25in}
\setlength{\textwidth}{6.25in}
\setlength{\topmargin}{-0.4in}
\setlength{\textheight}{8.5in}

\newcommand{\heading}[5]{
   \renewcommand{\thepage}{#1-\arabic{page}}
   \noindent
   \begin{center}
   \framebox{
      \vbox{
    \hbox to 6.2in { {\bf CS390 Computational Game Theory and Mechanism Design}
     	 \hfill #2 }
       \vspace{4mm}
       \hbox to 6.2in { {\Large \hfill #5  \hfill} }
       \vspace{2mm}
       \hbox to 6.2in { {\it #3 \hfill #4} }
      }
   }
   \end{center}
   \vspace*{4mm}
}

\newcommand{\handout}[3]{\heading{#1}{#2}{2012 ACM Class 5120309027 Huang Zen}{}{#3}}

\setlength{\parindent}{0in}
\setlength{\parskip}{0.1in}

\newenvironment{proof}{\noindent{\em Proof.} \hspace*{1mm}}{
\hspace*{\fill} $\Box$ }

\begin{document}
\handout{1}{July 21, 2013}{Problem Set 5}

\paragraph{Problem 1} (No collaborator.)
\\
\begin{proof}
The definition of \emph{lower contour} $L(o,v_i)$ ($\forall i\in N,o\subset O,
v_i\in\Theta_i$) is
$$L(o,v_i)=\lbrace o' \ \mathlarger{|} \ u_i(v_i,o)\geq u_i(v_i,o') \rbrace$$

Since the utility function does not have ties, so in this environment $E$
$$L(o,v_i)=\lbrace o' \ \mathlarger{|} \ u_i(v_i,o)>u_i(v_i,o') \rbrace$$

Consider $\forall\theta,\theta'\in\Theta$, $\forall o\in f^{CON}(\theta)$, 
and $\forall i\in N$, $L(o,\theta_i)\subseteq L(o,\theta_i')$,
 we have
$$\lbrace i \ \mathlarger{|} \ u_i(\theta_i,o)>u_i(\theta_i,o')\rbrace\subseteq
 \lbrace i \ \mathlarger{|} \ u_i(\theta_i',o)>u_i(\theta_i',o')\rbrace$$
$$
\mathlarger{\mathlarger{|}} \lbrace i \ \mathlarger{|} \ u_i(\theta_i,o)>u_i(\theta_i,o')\rbrace\mathlarger{\mathlarger{|}}
\leq\mathlarger{\mathlarger{|}} \lbrace i \ \mathlarger{|} \ u_i(\theta_i',o)>u_i(\theta_i',o')\rbrace\mathlarger{\mathlarger{|}}
$$ 
So $o\in f^{CON}(\theta')$ then $f^{CON}$ satisfies monotonicity.  
\end{proof}

\bigskip

\paragraph{Problem 2} (No collaborator.)
\\
$f^{BC}$ doesn't satisfy monotonicity.
\\
\begin{proof}
Suppose originally player 1 get $(3, 1)$ and player 2 get $(2, 2)$ and player 3 get $(1, 3)$.
After the change player 1 get $(3, 1)$, player 2 get $(1, 2)$ and player 3 get $(2, 3)$.
It satisfies the lower contour property,
but player 1 is not the winner any more.
So it is not monotone.
\end{proof}
\bigskip

\paragraph{Problem 3} (No collaborator.)
\\
$f(\theta)$ defined in this game is not monotone.
\\
\begin{proof}
Let $\theta=(2,1,1,...,1)$
and $\theta'=(0.5,1,1,...,1)$
and $o=(1,p)$ while $p=(0.5,0.5,...,0.5)$.
Obviously $o \in f(\theta)$ but $o \not\in f(\theta')$.
Next we will see that for all $i$, $L(o,\theta_i) \subset L(o,\theta'_i)$ equivalent to
$$\forall o', u_(\theta_i,o) \geq u_i(\theta_i,o') \Rightarrow u'_i(\theta_i,o) \geq u'_i(\theta_i,o')$$
which leads to the statement proved.
\\
For $i=2,3,...,n$ since
$\theta_i=\theta'_i$,
$u_i(\theta_i,o)=u_i(\theta'_i,o)$
and $u_i(\theta_i,o')=u_i(\theta'_i,o')$,
the above statement holds.
\\
For player 1, $u_1(\theta_1,o)=1.5$ and $u_1(\theta'_1,o)=0$.
Suppose in another $o'$ his price is $p_1$.
If in $o'$, player 1 still win the good, then $u_1(\theta_1,o')=2-p_1$ and  $u_1(\theta'_1,o')=0.5-p_1$.
When $p_1\geq 0.5$, $u_1(\theta_1,o') \leq 1.5$ and $u_1(\theta'_1,o') \leq 0$,
so that the above statement also holds.
\\
So the above statement holds for all $n$, which shows $f(\theta)$ is not monotone.
\end{proof}

\bibliographystyle{agsm}

\begin{thebibliography}{99}

\bibitem{OR94}{M. J. Osborne and A. Rubinstein. {\em A course in game theory.} MIT Press, 1994.}

\bibitem{NRTV07}{N. Nisan, T. Roughgarden, E. Tardos, and V. Vazirani (eds). {\em Algorithmic game theory.} Cambridge University Press, 2007. (Available at \url{http://www.cambridge.org/journals/nisan/downloads/Nisan_Non-printable.pdf}.)}

\end{thebibliography}

\end{document}








