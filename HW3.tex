\documentclass[12pt]{article}
\usepackage{amsfonts,amsmath,amssymb,graphicx,url}

% Old Stuff
%%\oddsidemargin=0.15in
%%\evensidemargin=0.15in
%%\topmargin=-.5in
%%\textheight=9in
%%\textwidth=6.25in

\setlength{\oddsidemargin}{.25in}
\setlength{\evensidemargin}{.25in}
\setlength{\textwidth}{6.25in}
\setlength{\topmargin}{-0.4in}
\setlength{\textheight}{8.5in}

\newcommand{\heading}[5]{
   \renewcommand{\thepage}{#1-\arabic{page}}
   \noindent
   \begin{center}
   \framebox{
      \vbox{
    \hbox to 6.2in { {\bf CS390 Computational Game Theory and Mechanism Design}
         \hfill #2 }
       \vspace{4mm}
       \hbox to 6.2in { {\Large \hfill #5  \hfill} }
       \vspace{2mm}
       \hbox to 6.2in { {\it #3 \hfill #4} }
      }
   }
   \end{center}
   \vspace*{4mm}
}

\newcommand{\handout}[3]{\heading{#1}{#2}{2012 ACM Class 5120309027 Huang Zen}{}{#3}}

\setlength{\parindent}{0in}
\setlength{\parskip}{0.1in}

\newenvironment{proof}{\noindent{\em Proof.} \hspace*{1mm}}{
\hspace*{\fill} $\Box$ }

\begin{document}
\handout{1}{July 13, 2013}{Problem Set 3}

\paragraph{Problem 1} (No collaborator.)
\\
The objective cost function when $x$ people choose the lower edge is 
$$c(x)=\frac{x(\frac{x}{n})^d+(n-x)}{n}, \ x\in[0,n]$$
To calculate the optimal cost, take the deriviation of $c(x)$ 
$$ c'(x)= \frac{- 1 + (d + 1)\frac{x^d}{n^d}}{n} $$
Solve for $c'(x)=0$ we have
$$ x=\frac{n}{(d+1)^{1/d}}$$
so at $\displaystyle x=\frac{n}{(d+1^{1/d})}$, $c(x)$ is optimal.
\\
Since $(x/n)^d\leq 1$, there are still $2$ equilibrium: $x=n$ and $x=n-1$. 
So
$$PoA=\frac{c(n)}{\min c(x)}
\mbox{ or }
\frac{c(n-1)}{\min c(x)}$$
When $n\to\infty$, $c(n)=c(n-1)=1+o(1)$, 
$\displaystyle PoA\to\frac{1}{1 - (d + 1)^{-1/d} + (d + 1)^{-1-1/d}}$.\\
When $d\to\infty$, 
$\displaystyle PoA\to+\infty$.


\bigskip

\paragraph{Problem 2} (No collaborator.)
\\
\begin{proof}
Let the strategy profile with the optimal sum of utility be $S$,
and the NE with the worst objective function be $S^\star$.\\
Consider one  player $i$, 
if his utility for $S^\star$ exceed n times his utility for $S$,
then he can change his strategy to that in $S$, 
since no matter how many people choose the same route as him,
his cost on this route won't exceed $n$ times any possible cost on this route. 
\\
This is true for any player, 
Then the total cost in NE won't exceed $n$ times that of the optimal objective function.
Therefore the price of anarchy won't exceed $1$.
\end{proof}

\bigskip

\paragraph{Problem 3} (No collaborator.)
\\
First, the pure-strategy Nash equilibria of this game are precisely 
the m! outcomes in which each player selects a distinct machine.
Second, mixed strategy $\sigma=(\sigma_i)$ where 
$\sigma_i=(\frac{1}{n},...\frac{1}{n})$ for each player $i$
is the unique mixed Nash equilibrium(\cite{NRTV07}. 17.2.3).
\\
let $X_{ij}$ denote the indicator random variable for 
the event that player $i$ selects the machine $j$.
If the first player selects machine $j$, then it incurs a cost of $\displaystyle 1+\sum_{\substack{i>1}}X_{ij}$. 
By linearity of expectation, its expected cost on this machine is $\displaystyle 1+\sum_{\substack{i>1}}E[X_{ij}]=2-\frac{1}{n}$(\cite{NRTV07}. Example 17.4).
\\
So the objective cost at the unique mixed NE is $\displaystyle 2-\frac{1}{n}$
and since the optimal objective function is equal to $1$, the price of anarchy is 
$$PoA=2-\frac{1}{n}$$

\bibliographystyle{agsm}

\begin{thebibliography}{99}

\bibitem{OR94}{M. J. Osborne and A. Rubinstein. {\em A course in game theory.} MIT Press, 1994.}

\bibitem{NRTV07}{N. Nisan, T. Roughgarden, E. Tardos, and V. Vazirani (eds). {\em Algorithmic game theory.} Cambridge University Press, 2007. (Available at \url{http://www.cambridge.org/journals/nisan/downloads/Nisan_Non-printable.pdf}.)}

\end{thebibliography}

\end{document}








